\begin{song}{Valomerkin jälkeen}

    \begin{alternatinglyrics}
        \emph{Solo}:    & 27.9.1968 annetun alkoholilain 47. pykälä\\
                        & pitää sisällään seuraavassa kuultavan\\
                        & karun totuuden:\\
        \emph{Tutti}:  & Anniskeluajan päättyminen\\
                        & on asiakkaille ilmoitettava\\
                        & niin sanotulla valomerkillä\\
                        & tahi muulla sopivalla tavalla.\\
                        & \repetitionbegin{} Valomerkin jälkeen saadaan asiakkaalle\\
                        & tarjoilla se alkoholijuoma-annos, joka\\
                        & on ennen anniskeluajan päättymistä\\
                        & tilattu ja otettu alkoholikassasta. \repetitionend{}\\
                        & Väkeviä juomia ja viinejä\\
                        & pulloittain tarjoiltaessa on\\
                        & pulloissa jäljellä olevat\\
                        & alkoholijuomat palautettava.\\
        \emph{Solo}:    & Kuitenkin valomerkin antamisen jälkeen\\
                        & saadaan asiakkaalle kaataa niin sanottu\\
                        & perusannos täyteen.\\
        \emph{Tutti}:  & Valomerkin jälkeen saadaan asiakkaalle\dots\\
    \end{alternatinglyrics}

\end{song}